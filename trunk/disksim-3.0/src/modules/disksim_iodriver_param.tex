\noindent 
\begin{tabular}{|p{1.5in}|p{3.5in}|p{0.5in}|p{0.5in}|}
\cline{1-4}
\texttt{disksim\_iodriver} & \texttt{type} & int & required \\ 
\cline{1-4}
\multicolumn{4}{|p{6in}|}{
This is included for extensibility purposes.
}\\ 
\cline{1-4}
\multicolumn{4}{p{5in}}{}\\
\end{tabular}\\ 
\noindent 
\begin{tabular}{|p{1.5in}|p{3.5in}|p{0.5in}|p{0.5in}|}
\cline{1-4}
\texttt{disksim\_iodriver} & \texttt{Constant access time} & float & required \\ 
\cline{1-4}
\multicolumn{4}{|p{6in}|}{
This specifies any of several forms of storage simulation abstraction.
A positive value indicates a fixed access time (after any queueing
delays) for each disk request. With this option, requests do not
propagate to lower levels of the storage subsystem (and the stats and
configuration of lower levels are therefore meaningless). $-1.0$
indicates that the trace provides a measured access time for each
request, which should be used instead of any simulated access times.
$-2.0$ indicates that the trace provides a measured queue time for
each request, which should be used instead of any simulated queue
times. (Note: This can cause problems if multiple requests are
simultaneously issued to to disks that don't support queueing.)
$-3.0$ indicates that the trace provides measured values for both the
access time and the queue time. Finally, $0.0$ indicates that the
simulation should compute all access and queue times as appropriate
given the changing state of the storage subsystem.
}\\ 
\cline{1-4}
\multicolumn{4}{p{5in}}{}\\
\end{tabular}\\ 
\noindent 
\begin{tabular}{|p{1.5in}|p{3.5in}|p{0.5in}|p{0.5in}|}
\cline{1-4}
\texttt{disksim\_iodriver} & \texttt{Use queueing in subsystem} & int & required \\ 
\cline{1-4}
\multicolumn{4}{|p{6in}|}{
This specifies whether or not the device driver allows more than one
request to be outstanding (in the storage subsystem) at any point in
time. During initialization, this parameter is combined with the
parameterized capabilities of the subsystem itself to determine
whether or not queueing in the subsystem is appropriate.
}\\ 
\cline{1-4}
\multicolumn{4}{p{5in}}{}\\
\end{tabular}\\ 
\noindent 
\begin{tabular}{|p{1.5in}|p{3.5in}|p{0.5in}|p{0.5in}|}
\cline{1-4}
\texttt{disksim\_iodriver} & \texttt{Scheduler} & block & required \\ 
\cline{1-4}
\multicolumn{4}{|p{6in}|}{
This is an ioqueue; see section \ref{param.queue}.
}\\ 
\cline{1-4}
\multicolumn{4}{p{5in}}{}\\
\end{tabular}\\ 
