\noindent 
\begin{tabular}{|p{1.5in}|p{3.5in}|p{0.5in}|p{0.5in}|}
\cline{1-4}
\texttt{disksim\_global} & \texttt{Init Seed} & int & optional \\ 
\cline{1-4}
\multicolumn{4}{|p{6in}|}{
This specifies the initial seed for the random number generator.
The initial seed value is applied at the very beginning of the
simulation and is used during the initialization phase (e.g.,~for
determining initial rotational positions). Explicitly specifying the
random generator seed enables experiment repeatability.
}\\ 
\cline{1-4}
\multicolumn{4}{p{5in}}{}\\
\end{tabular}\\ 
\noindent 
\begin{tabular}{|p{1.5in}|p{3.5in}|p{0.5in}|p{0.5in}|}
\cline{1-4}
\texttt{disksim\_global} & \texttt{Init Seed with time} & int & optional \\ 
\cline{1-4}
\multicolumn{4}{|p{6in}|}{
If a nonzero value is provided, DiskSim will use the current system
time to initialize the ``Init Seed'' parameter.
}\\ 
\cline{1-4}
\multicolumn{4}{p{5in}}{}\\
\end{tabular}\\ 
\noindent 
\begin{tabular}{|p{1.5in}|p{3.5in}|p{0.5in}|p{0.5in}|}
\cline{1-4}
\texttt{disksim\_global} & \texttt{Real Seed} & int & optional \\ 
\cline{1-4}
\multicolumn{4}{|p{6in}|}{
The `real' seed value is applied after the initialization phase
and is used during the simulation phase (e.g.,~for synthetic
workload generation). This allows multiple synthetic workloads
(with different simulation seeds) to be run on equivalent
configurations (i.e.,~with identical initial seeds,
as specified above).
}\\ 
\cline{1-4}
\multicolumn{4}{p{5in}}{}\\
\end{tabular}\\ 
\noindent 
\begin{tabular}{|p{1.5in}|p{3.5in}|p{0.5in}|p{0.5in}|}
\cline{1-4}
\texttt{disksim\_global} & \texttt{Real Seed with time} & int & optional \\ 
\cline{1-4}
\multicolumn{4}{|p{6in}|}{
If a nonzero value is provided, DiskSim will use the current system
time to initialize the ``Real Seed'' parameter.
}\\ 
\cline{1-4}
\multicolumn{4}{p{5in}}{}\\
\end{tabular}\\ 
\noindent 
\begin{tabular}{|p{1.5in}|p{3.5in}|p{0.5in}|p{0.5in}|}
\cline{1-4}
\texttt{disksim\_global} & \texttt{Statistic warm-up time} & float & optional \\ 
\cline{1-4}
\multicolumn{4}{|p{6in}|}{
This specifies the amount of simulated time after which the statistics
will be reset.
}\\ 
\cline{1-4}
\multicolumn{4}{p{5in}}{}\\
\end{tabular}\\ 
\noindent 
\begin{tabular}{|p{1.5in}|p{3.5in}|p{0.5in}|p{0.5in}|}
\cline{1-4}
\texttt{disksim\_global} & \texttt{Statistic warm-up IOs} & int & optional \\ 
\cline{1-4}
\multicolumn{4}{|p{6in}|}{
This specifies the number of I/Os after which the statistics will be reset.
}\\ 
\cline{1-4}
\multicolumn{4}{p{5in}}{}\\
\end{tabular}\\ 
\noindent 
\begin{tabular}{|p{1.5in}|p{3.5in}|p{0.5in}|p{0.5in}|}
\cline{1-4}
\texttt{disksim\_global} & \texttt{Stat definition file} & string & required \\ 
\cline{1-4}
\multicolumn{4}{|p{6in}|}{
This specifies the name of the input file containing the
specifications for the statistical distributions to collect. This
file allows the user to control the number and sizes of histogram bins
into which data are collected. This file is mandatory.
Section~\ref{output.statdefs} describes its use.
}\\ 
\cline{1-4}
\multicolumn{4}{p{5in}}{}\\
\end{tabular}\\ 
\noindent 
\begin{tabular}{|p{1.5in}|p{3.5in}|p{0.5in}|p{0.5in}|}
\cline{1-4}
\texttt{disksim\_global} & \texttt{Output file for trace of I/O requests simulated} & string & optional \\ 
\cline{1-4}
\multicolumn{4}{|p{6in}|}{
This specifies the name of the output file to contain a trace of disk
request arrivals (in the default ASCII trace format described in
Section~\ref{input.traces}). This allows instances of synthetically
generated workloads to be saved and analyzed after the simulation
completes. This is particularly useful for analyzing (potentially
pathological) workloads produced by a system-level model.
}\\ 
\cline{1-4}
\multicolumn{4}{p{5in}}{}\\
\end{tabular}\\ 
