\noindent 
\begin{tabular}{|p{1.5in}|p{3.5in}|p{0.5in}|p{0.5in}|}
\cline{1-4}
\texttt{disksim\_iomap} & \texttt{tracedev} & int & required \\ 
\cline{1-4}
\multicolumn{4}{|p{6in}|}{
This specifies the traced device affected by this mapping.
}\\ 
\cline{1-4}
\multicolumn{4}{p{5in}}{}\\
\end{tabular}\\ 
\noindent 
\begin{tabular}{|p{1.5in}|p{3.5in}|p{0.5in}|p{0.5in}|}
\cline{1-4}
\texttt{disksim\_iomap} & \texttt{simdev} & string & required \\ 
\cline{1-4}
\multicolumn{4}{|p{6in}|}{
This specifies the simulated device such requests should access.
}\\ 
\cline{1-4}
\multicolumn{4}{p{5in}}{}\\
\end{tabular}\\ 
\noindent 
\begin{tabular}{|p{1.5in}|p{3.5in}|p{0.5in}|p{0.5in}|}
\cline{1-4}
\texttt{disksim\_iomap} & \texttt{locScale} & int & required \\ 
\cline{1-4}
\multicolumn{4}{|p{6in}|}{
This specifies a value by which a traced disk request location is multiplied to
generate the starting location (in bytes) of the simulated disk
request. For example, if the input trace specifies locations in terms
of 512-byte sectors, a value of 512 would result in an equivalent
logical space of requests.
}\\ 
\cline{1-4}
\multicolumn{4}{p{5in}}{}\\
\end{tabular}\\ 
\noindent 
\begin{tabular}{|p{1.5in}|p{3.5in}|p{0.5in}|p{0.5in}|}
\cline{1-4}
\texttt{disksim\_iomap} & \texttt{sizeScale} & int & required \\ 
\cline{1-4}
\multicolumn{4}{|p{6in}|}{
This specifies a value by which a traced disk request size is
multiplied to generate the size (in bytes) of the simulated disk
request.
}\\ 
\cline{1-4}
\multicolumn{4}{p{5in}}{}\\
\end{tabular}\\ 
\noindent 
\begin{tabular}{|p{1.5in}|p{3.5in}|p{0.5in}|p{0.5in}|}
\cline{1-4}
\texttt{disksim\_iomap} & \texttt{offset} & int & optional \\ 
\cline{1-4}
\multicolumn{4}{|p{6in}|}{
This specifies a value to be added to each simulated request's
starting location. This is especially useful for combining multiple
trace devices' logical space into the space of a single simulated
device.
}\\ 
\cline{1-4}
\multicolumn{4}{p{5in}}{}\\
\end{tabular}\\ 
