
\subsection{Model Configuration}
Diskmodel uses libparam to input the following blocks of parameter data:

\begin{verbatim}
dm_disk
dm_layout_g1
dm_layout_g1_zone
dm_mech_g1
\end{verbatim}

\subsubsection{dm\_disk}

The outer \texttt{dm\_disk} block contains the top-level parameters
which are used to fill in the \texttt{dm\_disk\_if} structure.  The
only valid value for ``Layout Model'' is a \texttt{dm\_layout\_g1}
block and for ``Mechanical Model,'' a \texttt{dm\_mech\_g1} block.\\

\noindent 
\begin{tabular}{|p{1.5in}|p{3.5in}|p{0.5in}|p{0.5in}|}
\cline{1-4}
\texttt{dm\_disk} & \texttt{Block count} & int & required \\ 
\cline{1-4}
\multicolumn{4}{|p{6in}|}{
This specifies the number of data blocks. This capacity is exported by the
disk (e.g.,~to a disk array controller). It is not used directly
during simulation, but is compared to a similar value computed from
other disk parameters. A warning is reported if the values differ.
}\\ 
\cline{1-4}
\multicolumn{4}{p{5in}}{}\\
\end{tabular}\\ 
\noindent 
\begin{tabular}{|p{1.5in}|p{3.5in}|p{0.5in}|p{0.5in}|}
\cline{1-4}
\texttt{dm\_disk} & \texttt{Number of data surfaces} & int & required \\ 
\cline{1-4}
\multicolumn{4}{|p{6in}|}{
This specifies the number of magnetic media surfaces (not platters!) on
which data are recorded. Dedicated servo surfaces should not be
counted for this parameter.
}\\ 
\cline{1-4}
\multicolumn{4}{p{5in}}{}\\
\end{tabular}\\ 
\noindent 
\begin{tabular}{|p{1.5in}|p{3.5in}|p{0.5in}|p{0.5in}|}
\cline{1-4}
\texttt{dm\_disk} & \texttt{Number of cylinders} & int & required \\ 
\cline{1-4}
\multicolumn{4}{|p{6in}|}{
This specifies the number of physical cylinders. All cylinders that impact
the logical to physical mappings should be included.
}\\ 
\cline{1-4}
\multicolumn{4}{p{5in}}{}\\
\end{tabular}\\ 
\noindent 
\begin{tabular}{|p{1.5in}|p{3.5in}|p{0.5in}|p{0.5in}|}
\cline{1-4}
\texttt{dm\_disk} & \texttt{Mechanical Model} & block & required \\ 
\cline{1-4}
\multicolumn{4}{|p{6in}|}{
This block defines the disk's mechanical model. Currently,
the only available implementation is \texttt{dm\_mech\_g1}.
}\\ 
\cline{1-4}
\multicolumn{4}{p{5in}}{}\\
\end{tabular}\\ 
\noindent 
\begin{tabular}{|p{1.5in}|p{3.5in}|p{0.5in}|p{0.5in}|}
\cline{1-4}
\texttt{dm\_disk} & \texttt{Layout Model} & block & required \\ 
\cline{1-4}
\multicolumn{4}{|p{6in}|}{
This block defines the disk's layout model. Currently,
the only available implementation is \texttt{dm\_layout\_g1}.
}\\ 
\cline{1-4}
\multicolumn{4}{p{5in}}{}\\
\end{tabular}\\ 


\subsubsection{dm\_layout\_g1}

The \texttt{dm\_layout\_g1} block provides parameters for a first
generation (g1) layout model.\\

\noindent 
\begin{tabular}{|p{1.5in}|p{3.5in}|p{0.5in}|p{0.5in}|}
\cline{1-4}
\texttt{dm\_layout\_g1} & \texttt{LBN-to-PBN mapping scheme} & int & required \\ 
\cline{1-4}
\multicolumn{4}{|p{6in}|}{
This specifies the type of LBN-to-PBN mapping used by the disk.
0 indicates that the conventional mapping scheme is used:
LBNs advance along the 0th track of the 0th cylinder, then along the
1st track of the 0th cylinder, thru the end of the 0th cylinder, then
to the 0th track of the 1st cylinder, and so forth.
1 indicates that the conventional mapping scheme is modified slightly,
such that cylinder switches do not involve head switches. Thus, after
LBNs are assigned to the last track of the 0th cylinder, they are
assigned to the last track of the 1st cylinder, the next-to-last track
of the 1st cylinder, thru the 0th track of the 1st cylinder. LBNs are
then assigned to the 0th track of the 2nd cylinder, and so on
(``first cylinder is normal'').
2 is like 1 except that the serpentine pattern does not reset at the
beginning of each zone; rather, even cylinders are always ascending and
odd cylinders are always descending.
}\\ 
\cline{1-4}
\multicolumn{4}{p{5in}}{}\\
\end{tabular}\\ 
\noindent 
\begin{tabular}{|p{1.5in}|p{3.5in}|p{0.5in}|p{0.5in}|}
\cline{1-4}
\texttt{dm\_layout\_g1} & \texttt{Sparing scheme used} & int & required \\ 
\cline{1-4}
\multicolumn{4}{|p{6in}|}{
This specifies the type of sparing used by the disk. Later parameters determine
where spare space is allocated.
0~indicates that no spare sectors are allocated.
1~indicates that entire tracks of spare sectors are allocated at the ``end''
of some or all zones (sets of cylinders).
2~indicates that spare sectors are allocated at the ``end'' of each cylinder.
3~indicates that spare sectors are allocated at the ``end'' of each track.
4~indicates that spare sectors are allocated at the ``end'' of each cylinder
and that slipped sectors do not utilize these spares (more spares are located
at the ``end'' of the disk).
5~indicates that spare sectors are allocated at the ``front'' of each cylinder.
6~indicates that spare sectors are allocated at the ``front'' of each cylinder
and that slipped sectors do not utilize these spares (more spares are located
at the ``end'' of the disk).
7~indicates that spare sectors are allocated at the ``end'' of the disk.
8~indicates that spare sectors are allocated at the ``end'' of each range
of cylinders.
9~indicates that spare sectors are allocated at the ``end'' of each zone.
10~indicates that spare sectors are allocated at the ``end'' of each zone
and that slipped sectors do not use these spares (more spares are located
at the ``end'' of the disk).
}\\ 
\cline{1-4}
\multicolumn{4}{p{5in}}{}\\
\end{tabular}\\ 
\noindent 
\begin{tabular}{|p{1.5in}|p{3.5in}|p{0.5in}|p{0.5in}|}
\cline{1-4}
\texttt{dm\_layout\_g1} & \texttt{Rangesize for sparing} & int & required \\ 
\cline{1-4}
\multicolumn{4}{|p{6in}|}{
This specifies the range (e.g., of cylinders) over which spares are
allocated and maintained. Currently, this value is relevant only for
disks that use ``sectors per cylinder range'' sparing schemes.
}\\ 
\cline{1-4}
\multicolumn{4}{p{5in}}{}\\
\end{tabular}\\ 
\noindent 
\begin{tabular}{|p{1.5in}|p{3.5in}|p{0.5in}|p{0.5in}|}
\cline{1-4}
\texttt{dm\_layout\_g1} & \texttt{Skew units} & string & optional \\ 
\cline{1-4}
\multicolumn{4}{|p{6in}|}{
This sets the units with which units are input: \texttt{revolutions} or
\texttt{sectors}. The ``disk-wide'' value set here may be overridden
per-zone. The default unit is \texttt{sectors}.
}\\ 
\cline{1-4}
\multicolumn{4}{p{5in}}{}\\
\end{tabular}\\ 
\noindent 
\begin{tabular}{|p{1.5in}|p{3.5in}|p{0.5in}|p{0.5in}|}
\cline{1-4}
\texttt{dm\_layout\_g1} & \texttt{Zones} & list & required \\ 
\cline{1-4}
\multicolumn{4}{|p{6in}|}{
This is a list of zone block values describing the zones/bands of the disk.
}\\ 
\cline{1-4}
\multicolumn{4}{p{5in}}{}\\
\end{tabular}\\ 


The \texttt{Zones} parameter is a list of zone blocks each of which
contains the following fields:\\

\noindent 
\begin{tabular}{|p{1.5in}|p{3.5in}|p{0.5in}|p{0.5in}|}
\cline{1-4}
\texttt{dm\_layout\_g1\_zone} & \texttt{First cylinder number} & int & required \\ 
\cline{1-4}
\multicolumn{4}{|p{6in}|}{
This specifies the first physical cylinder in the zone.
}\\ 
\cline{1-4}
\multicolumn{4}{p{5in}}{}\\
\end{tabular}\\ 
\noindent 
\begin{tabular}{|p{1.5in}|p{3.5in}|p{0.5in}|p{0.5in}|}
\cline{1-4}
\texttt{dm\_layout\_g1\_zone} & \texttt{Last cylinder number} & int & required \\ 
\cline{1-4}
\multicolumn{4}{|p{6in}|}{
This specifies the last physical cylinder in the zone.
}\\ 
\cline{1-4}
\multicolumn{4}{p{5in}}{}\\
\end{tabular}\\ 
\noindent 
\begin{tabular}{|p{1.5in}|p{3.5in}|p{0.5in}|p{0.5in}|}
\cline{1-4}
\texttt{dm\_layout\_g1\_zone} & \texttt{Blocks per track} & int & required \\ 
\cline{1-4}
\multicolumn{4}{|p{6in}|}{
This specifies the number of sectors (independent of logical-to-physical
mappings) on each physical track in the zone.
}\\ 
\cline{1-4}
\multicolumn{4}{p{5in}}{}\\
\end{tabular}\\ 
\noindent 
\begin{tabular}{|p{1.5in}|p{3.5in}|p{0.5in}|p{0.5in}|}
\cline{1-4}
\texttt{dm\_layout\_g1\_zone} & \texttt{Offset of first block} & float & required \\ 
\cline{1-4}
\multicolumn{4}{|p{6in}|}{
This specifies the physical offset of the first logical sector in the
zone. Physical sector 0 of every track is assumed to begin at the
same angle of rotation. This may be in either sectors or revolutions
according to the ``Skew units'' parameter.
}\\ 
\cline{1-4}
\multicolumn{4}{p{5in}}{}\\
\end{tabular}\\ 
\noindent 
\begin{tabular}{|p{1.5in}|p{3.5in}|p{0.5in}|p{0.5in}|}
\cline{1-4}
\texttt{dm\_layout\_g1\_zone} & \texttt{Skew units} & string & optional \\ 
\cline{1-4}
\multicolumn{4}{|p{6in}|}{
Default is \texttt{sectors}. This value overrides any set in the
surrounding layout block.
}\\ 
\cline{1-4}
\multicolumn{4}{p{5in}}{}\\
\end{tabular}\\ 
\noindent 
\begin{tabular}{|p{1.5in}|p{3.5in}|p{0.5in}|p{0.5in}|}
\cline{1-4}
\texttt{dm\_layout\_g1\_zone} & \texttt{Empty space at zone front} & int & required \\ 
\cline{1-4}
\multicolumn{4}{|p{6in}|}{
This specifies the size of the ``management area'' allocated at the
beginning of the zone for internal data structures. This area can not
be accessed during normal activity and is not part of the disk's
logical-to-physical mapping.
}\\ 
\cline{1-4}
\multicolumn{4}{p{5in}}{}\\
\end{tabular}\\ 
\noindent 
\begin{tabular}{|p{1.5in}|p{3.5in}|p{0.5in}|p{0.5in}|}
\cline{1-4}
\texttt{dm\_layout\_g1\_zone} & \texttt{Skew for track switch} & float & optional \\ 
\cline{1-4}
\multicolumn{4}{|p{6in}|}{
This specifies the number of physical sectors that are skipped when
assigning logical block numbers to physical sectors at a track
crossing point. Track skew is computed by the manufacturer to
optimize sequential access. This may be in either sectors or
revolutions according to the ``Skew units'' parameter.
}\\ 
\cline{1-4}
\multicolumn{4}{p{5in}}{}\\
\end{tabular}\\ 
\noindent 
\begin{tabular}{|p{1.5in}|p{3.5in}|p{0.5in}|p{0.5in}|}
\cline{1-4}
\texttt{dm\_layout\_g1\_zone} & \texttt{Skew for cylinder switch} & float & optional \\ 
\cline{1-4}
\multicolumn{4}{|p{6in}|}{
This specifies the number of physical sectors that are skipped when
assigning logical block numbers to physical sectors at a cylinder
crossing point. Cylinder skew is computed by the manufacturer to
optimize sequential access. This may be in either sectors or
revolutions according to the ``Skew units'' parameter.
}\\ 
\cline{1-4}
\multicolumn{4}{p{5in}}{}\\
\end{tabular}\\ 
\noindent 
\begin{tabular}{|p{1.5in}|p{3.5in}|p{0.5in}|p{0.5in}|}
\cline{1-4}
\texttt{dm\_layout\_g1\_zone} & \texttt{Number of spares} & int & required \\ 
\cline{1-4}
\multicolumn{4}{|p{6in}|}{
This specifies the number of spare storage locations -- sectors or tracks,
depending on the sparing scheme chosen -- allocated per region of
coverage which may be a track, cylinder, or zone, depending on the
sparing scheme. For example, if the sparing scheme is 1, indicating
that spare tracks are allocated at the end of the zone, the value of
this parameter indicates how many spare tracks have been allocated for
this zone.
}\\ 
\cline{1-4}
\multicolumn{4}{p{5in}}{}\\
\end{tabular}\\ 
\noindent 
\begin{tabular}{|p{1.5in}|p{3.5in}|p{0.5in}|p{0.5in}|}
\cline{1-4}
\texttt{dm\_layout\_g1\_zone} & \texttt{slips} & list & required \\ 
\cline{1-4}
\multicolumn{4}{|p{6in}|}{
This is a list of lbns for previously detected defective media
locations -- sectors or tracks, depending upon the sparing scheme
chosen -- that were skipped-over or ``slipped'' when the
logical-to-physical mapping was last created. Each integer in the
list indicates the slipped (defective) location.
}\\ 
\cline{1-4}
\multicolumn{4}{p{5in}}{}\\
\end{tabular}\\ 
\noindent 
\begin{tabular}{|p{1.5in}|p{3.5in}|p{0.5in}|p{0.5in}|}
\cline{1-4}
\texttt{dm\_layout\_g1\_zone} & \texttt{defects} & list & required \\ 
\cline{1-4}
\multicolumn{4}{|p{6in}|}{
This list describes previously detected defective media
locations -- sectors or tracks, depending upon the sparing scheme
chosen -- that have been remapped to alternate physical locations.
The elements of the list are interpreted as pairs wherein the first
number is the original (defective) location and the second number
indicates the replacement location. Note that these locations
will both be either a physical sector number or a physical track
number, depending on the sparing scheme chosen.
}\\ 
\cline{1-4}
\multicolumn{4}{p{5in}}{}\\
\end{tabular}\\ 


\subsubsection{dm\_mech\_g1}

The \texttt{dm\_mech\_g1} block provides parameters for a first
generation (g1) mechanical model.\\

\noindent 
\begin{tabular}{|p{1.5in}|p{3.5in}|p{0.5in}|p{0.5in}|}
\cline{1-4}
\texttt{dm\_mech\_g1} & \texttt{Access time type} & string & required \\ 
\cline{1-4}
\multicolumn{4}{|p{6in}|}{
This specifies the method for computing mechanical delays. Legal values
are \texttt{constant} which indicates a fixed per-request access time
(i.e.,~actual mechanical activity is not modeled),
\texttt{averageRotation} which indicates that seek activity should be
modeled but rotational latency is assumed to be equal to one half of
a rotation (the statistical mean for random disk access) and
\texttt{trackSwitchPlusRotation} which indicates that both seek and
rotational activity should be modeled.
}\\ 
\cline{1-4}
\multicolumn{4}{p{5in}}{}\\
\end{tabular}\\ 
\noindent 
\begin{tabular}{|p{1.5in}|p{3.5in}|p{0.5in}|p{0.5in}|}
\cline{1-4}
\texttt{dm\_mech\_g1} & \texttt{Constant access time} & float & optional \\ 
\cline{1-4}
\multicolumn{4}{|p{6in}|}{
Provides the constant access time to be used if the access time type
is set to constant.
}\\ 
\cline{1-4}
\multicolumn{4}{p{5in}}{}\\
\end{tabular}\\ 
\noindent 
\begin{tabular}{|p{1.5in}|p{3.5in}|p{0.5in}|p{0.5in}|}
\cline{1-4}
\texttt{dm\_mech\_g1} & \texttt{Seek type} & string & required \\ 
\cline{1-4}
\multicolumn{4}{|p{6in}|}{
This specifies the method for computing seek delays.
Legal values are the following:
\texttt{linear} indicates that the single-cylinder seek time, the average
seek time, and the full-strobe seek time parameters should be used to
compute the seek time via linear interpolation.
\texttt{curve} indicates that the same three parameters should be used
with the seek equation described in \cite{Lee93} (see Section
\ref{seek.lee}).
\texttt{constant} indicates a fixed per-request seek time. The
\texttt{Constant seek time} parameter must be provided.
\texttt{hpl} indicates that the six-value \texttt{HPL seek equation values}
parameter (see below) should be used with the seek equation described
in \cite{Ruemmler94} (see below).
\texttt{hplplus10} indicates that the six-value \texttt{HPL seek
equation values} parameter (see below) should be used with the seek
equation described in \cite{Ruemmler94} for all seeks greater than
10~cylinders in length. For smaller seeks, use the 10-value
\texttt{First ten seek times} parameter (see below) as in
\cite{Worthington94}.
\texttt{extracted} indicates that a more complete seek curve (provided
in a separate file) should be used, with linear interpolation used to
compute the seek time for unspecified distances. If
\texttt{extracted} layout is used, the parameter \texttt{Full seek curve}
(below) must be provided.
}\\ 
\cline{1-4}
\multicolumn{4}{p{5in}}{}\\
\end{tabular}\\ 
\noindent 
\begin{tabular}{|p{1.5in}|p{3.5in}|p{0.5in}|p{0.5in}|}
\cline{1-4}
\texttt{dm\_mech\_g1} & \texttt{Average seek time} & float & optional \\ 
\cline{1-4}
\multicolumn{4}{|p{6in}|}{
The mean time necessary to perform a random seek
}\\ 
\cline{1-4}
\multicolumn{4}{p{5in}}{}\\
\end{tabular}\\ 
\noindent 
\begin{tabular}{|p{1.5in}|p{3.5in}|p{0.5in}|p{0.5in}|}
\cline{1-4}
\texttt{dm\_mech\_g1} & \texttt{Constant seek time} & float & optional \\ 
\cline{1-4}
\multicolumn{4}{|p{6in}|}{
For the ``constant'' seek type (above).
}\\ 
\cline{1-4}
\multicolumn{4}{p{5in}}{}\\
\end{tabular}\\ 
\noindent 
\begin{tabular}{|p{1.5in}|p{3.5in}|p{0.5in}|p{0.5in}|}
\cline{1-4}
\texttt{dm\_mech\_g1} & \texttt{Single cylinder seek time} & float & optional \\ 
\cline{1-4}
\multicolumn{4}{|p{6in}|}{
This specifies the time necessary to seek to an adjacent cylinder.
}\\ 
\cline{1-4}
\multicolumn{4}{p{5in}}{}\\
\end{tabular}\\ 
\noindent 
\begin{tabular}{|p{1.5in}|p{3.5in}|p{0.5in}|p{0.5in}|}
\cline{1-4}
\texttt{dm\_mech\_g1} & \texttt{Full strobe seek time} & float & optional \\ 
\cline{1-4}
\multicolumn{4}{|p{6in}|}{
This specifies the full-strobe seek time (i.e.,~the time to seek from the
innermost cylinder to the outermost cylinder).
}\\ 
\cline{1-4}
\multicolumn{4}{p{5in}}{}\\
\end{tabular}\\ 
\noindent 
\begin{tabular}{|p{1.5in}|p{3.5in}|p{0.5in}|p{0.5in}|}
\cline{1-4}
\texttt{dm\_mech\_g1} & \texttt{Full seek curve} & string & optional \\ 
\cline{1-4}
\multicolumn{4}{|p{6in}|}{
The name of the input file containing the seek curve data.
The format of this file is described below.
}\\ 
\cline{1-4}
\multicolumn{4}{p{5in}}{}\\
\end{tabular}\\ 
\noindent 
\begin{tabular}{|p{1.5in}|p{3.5in}|p{0.5in}|p{0.5in}|}
\cline{1-4}
\texttt{dm\_mech\_g1} & \texttt{Add. write settling delay} & float & required \\ 
\cline{1-4}
\multicolumn{4}{|p{6in}|}{
This specifies the additional time required to precisely settle the
read/write head for writing (after a seek or head switch). As this
parameter implies, the seek times computed using the above parameter
values are for read access.
}\\ 
\cline{1-4}
\multicolumn{4}{p{5in}}{}\\
\end{tabular}\\ 
\noindent 
\begin{tabular}{|p{1.5in}|p{3.5in}|p{0.5in}|p{0.5in}|}
\cline{1-4}
\texttt{dm\_mech\_g1} & \texttt{Head switch time} & float & required \\ 
\cline{1-4}
\multicolumn{4}{|p{6in}|}{
This specifies the time required for a head switch (i.e.,~activating a
different read/write head in order to access a different media
surface).
}\\ 
\cline{1-4}
\multicolumn{4}{p{5in}}{}\\
\end{tabular}\\ 
\noindent 
\begin{tabular}{|p{1.5in}|p{3.5in}|p{0.5in}|p{0.5in}|}
\cline{1-4}
\texttt{dm\_mech\_g1} & \texttt{Rotation speed (in rpms)} & int & required \\ 
\cline{1-4}
\multicolumn{4}{|p{6in}|}{
This specifies the rotation speed of the disk platters in rpms.
}\\ 
\cline{1-4}
\multicolumn{4}{p{5in}}{}\\
\end{tabular}\\ 
\noindent 
\begin{tabular}{|p{1.5in}|p{3.5in}|p{0.5in}|p{0.5in}|}
\cline{1-4}
\texttt{dm\_mech\_g1} & \texttt{Percent error in rpms} & float & required \\ 
\cline{1-4}
\multicolumn{4}{|p{6in}|}{
This specifies the maximum deviation in the rotation speed specified
above. During initialization, the rotation speed for each
disk is randomly chosen from a uniform distribution of the specified
rotation speed $\pm$ the maximum allowed error.
%(as computed from this parameter's value).
This feature may be deprecated and should be avoided.
}\\ 
\cline{1-4}
\multicolumn{4}{p{5in}}{}\\
\end{tabular}\\ 
\noindent 
\begin{tabular}{|p{1.5in}|p{3.5in}|p{0.5in}|p{0.5in}|}
\cline{1-4}
\texttt{dm\_mech\_g1} & \texttt{First ten seek times} & list & optional \\ 
\cline{1-4}
\multicolumn{4}{|p{6in}|}{
This is a list of ten floating-point numbers specifying the seek time for seek
distances of 1~through 10~cylinders.
}\\ 
\cline{1-4}
\multicolumn{4}{p{5in}}{}\\
\end{tabular}\\ 
\noindent 
\begin{tabular}{|p{1.5in}|p{3.5in}|p{0.5in}|p{0.5in}|}
\cline{1-4}
\texttt{dm\_mech\_g1} & \texttt{HPL seek equation values} & list & optional \\ 
\cline{1-4}
\multicolumn{4}{|p{6in}|}{
This is a list containing six numbers specifying the variables
$V_1$ through $V_6$ of the seek equation described in \cite{Ruemmler94}
(see below).
}\\ 
\cline{1-4}
\multicolumn{4}{p{5in}}{}\\
\end{tabular}\\ 



\noindent\textbf{Lee's Seek Equation}
\label{seek.lee}

\begin{math}
seekTime(x) = \left\{ \begin{array} {r@{\quad:\quad}l}
              0 & if x = 0 \\
              a\sqrt{x-1} + b(x-1) + c & if x > 0
              \end{array} \right., \mbox{where} \\
\\
\indent
x \quad\mbox{is the seek distance in cylinders,} \\
\indent
a = (-10 minSeek + 15 avgSeek - 5 maxSeek) / (3 \sqrt{numCyl}), \\
\indent
b = (7 minSeek - 15 avgSeek + 8 maxSeek) / (3 numCyl), \mbox{and}\\
\indent
c = minSeek.
\end{math}\\


\noindent\textbf{The HPL Seek Equation}
\label{seek.hpl}

\begin{tabular}{cc}
Seek distance     & Seek time \\ \hline
1 cylinder        & $V_6$ \\
$<$$V_1$ cylinders  & $V_2$ + $V_3$ * $\sqrt{dist}$ \\
$>=$$V_1$ cylinders & $V_4$ + $V_5$ * dist \\
\end{tabular}
, where {\it dist} is the seek distance in cylinders.
\newline
If $V_6 == -1$, single-cylinder seeks are computed using the second equation.
$V_1$ is specified in cylinders, and $V_2$ through $V_6$ are specified in 
milliseconds.

$V_1$ must be a non-negative integer, $V_2 \ldots V_5$ must be
non-negative floats and $V_6$ must be either a non-negative float or
$-1$.\\


\noindent\textbf{Format of an extracted seek curve}
\label{seek.extract}

An extracted seek file contains a number of (seek-time,seek-distance)
data points.
The format of such a file is very simple: the first line is
\[\texttt{Seek distances measured: <n>}\]
where \texttt{<n>} is the number of seek distances provided in the
curve.  This line is followed by \texttt{<n>} lines of the form
\texttt{<distance>, <time>} where \texttt{<distance>} is the seek
distance measured in cylinders, and \texttt{<time>} is the amount of
time the seek took in milliseconds. e.g.

\begin{verbatim}
Seek distances measured: 4
1,  1.2
2,  1.5
5,  5
10, 9.2
\end{verbatim}




